% Une ligne commentaire débute par le caractère « % »

\documentclass[a4paper]{article}

% Options possibles : 10pt, 11pt, 12pt (taille de la fonte)
%                     oneside, twoside (recto simple, recto-verso)
%                     draft, final (stade de développement)

\usepackage[utf8]{inputenc}   % LaTeX, comprends les accents !
\usepackage[T1]{fontenc}      % Police contenant les caractères français
\usepackage[francais]{babel}  


\usepackage[a4paper,left=2cm,right=2cm]{geometry}% Format de la page, réduction des marges
\usepackage{graphicx}  % pour inclure des images

%\pagestyle{headings}        % Pour mettre des entêtes avec les titres
                              % des sections en haut de page

 \title{  Qui est-ce ?\\         % Les paramètres du titre : titre, auteur, date
  Projet de programmation}          
\author{Groupe \emph{AB}\\
  \emph{Martinet Hugo, Xihao Wang, Ziyu Gao}\\
  \emph{https://github.com/m-hugo/GuessWho}\\
  L2 informatique\\
  Faculté des Sciences\\
Université de Montpellier.}
\date{\today}             


\begin{document}

\maketitle                    % Faire un titre utilisant les données
                              % passées à \title, \author et \date

\begin{center}               % pour centrer 
  \includegraphics[scale=1]{rainette2.jpg}   % insertion d'une image
\end{center}

\begin{abstract}     % Résumé du travail

  \emph{Description très succinte du problème et des différentes étapes de réalisation}

\end{abstract}


\section{ Technologies utilisées  et organisation} % Commencer une section, etc.

\emph{Une page.}

\subsection{Choix du langage}         % Section plus petite

\begin{itemize}            % Liste non numérotée
\item
  Le language rust est utilisé pour l'integralité du projet
\item
   La bibliotheque graphique fltk-rs est utilisée pour l'affichage de la fenetre et des widgets, la fonctionalité de gestion des evenements par message incluse avec les widgets est très pratique\\\\
   La blibliothèque graphique egui à été fortement considérée et un prototype à été crée mais le concept d'"immediate mode" n'était pas facile à utiliser quand on fait une application qui doit initialiser beaucoup de donnés pendat l'execution, garder beaucoup d'états en memoire et avoir des widgets qui en influencent beaucoup d'autres 
\item 
   Le framework rocket est utilisé pour avoir un serveur similaire à node.js (mais avec la vitesse de rust!) pour le mode multijoueur
\item
	La blibliotheque reqwest est utilisée coté client pour acceder au serveur
\item
	La bibliotheque rand est utilisée pour tirer un personage aléatoire, le tirage de nombres aléatoires n'etant pas present dans la bibliotheque standard de rust
\item
	La bibliotheque serde est utilisée pour importer et exporter le json (planches et sauvegarde)
   
\end{itemize}

\subsection{Organisation du travail}

\begin{enumerate}           % Liste numérotée
\item
  Répartition du travail au sein du groupe\\
  J'ai commencé à faire le prototype tout seul avant de trouver un groupe puis j'ai travaillé avec Xihao Wang, Ziyu Gao à partir de la première séance, nous avons fait beaucoup de code ensemble mais ils ont aussi fait du code tout les 2, publié sous xihao-wang, et j'ai aussi fait pas mal de code moi même, publié comme le travail de groupe sous m-hugo\\
  J'ai fait l'architecture des programmes ainsi que le choix des bibliotheques, les deux parties les plus compliquées quand on fait un projet en rust car une bonne architecture nous permet de rendre le programme beaucoup plus interactif en evitant les problemes de "borrowing", d'"ownership" et de durée de vie des variables.
  Les bibliotheques rust sont très nombreuse et la plupart ne marchent pas ou manquent de fonctionalités 
\item
  Rythme de travail, mode de fonctionnement ...
  A la première seance je leur ai appris les bases du rust, nous avons ensuite travaillé ensemble 2-3 heures par semaine le lundi et nous finissions certaines taches de notre coté pendant la semaine
\end{enumerate}
\section{Étape 1 : permettre à l'utilisateur de jouer}

\emph{3 pages}

\begin{itemize}         
\item
  Fonctionnalités de l'application : interactions possibles de
  l'utilisateur.
\item
  Format du fichier JSON, contraintes éventuelles.\\
  \begin{verbatim}
  {
  "attrs": {
  	"Nom":["nom1", "nom2"], 
  	"attribut1":["possibilité1", "possibilité2"],
  	"attribut2":["possibilité1"]
  },
  "liste":[
    {
      "Nom":"nom1",
      "attribut1": "possibilité1",
      "attribut2": "possibilité1",
      "image": "./personnages/nom1.png"
    },
    {
      "Nom":"nom2",
      "attribut1": "possibilité2",
      "attribut2": "possibilité1",
      "image": "./personnages/nom2.png"
    },
    (22 autres "{}")
  ]
}

\end{verbatim}
 "attrs" contient tout les attributs et toutes les possibilités pour chaque attribut "liste" contient 24 blocs, chacun contenant tous les attributs definis dans "attrs" suivi de "image" avec le chemin de l'image depuis ce dossier ou un chemin absolu
\newline
\item
  Description de la forme des requêtes traitées.
\item
  Description des structures de données, classes, variables utilisées.
\item
  Description du traitement effectué lors d'une requête.
\end{itemize}


\section{Étape 2 : aider  à la saisie  des personnages}

\emph{2 pages}

\begin{itemize}    
\item
  Description du problème : format des données et du résultat.
\item
  Scénario des interactions avec l'utilisateur.\\
  Nous avions prevu de permettre aux utilisateurs de valider les attributs/valeurs avec la touche entrée pour pouvoir en saisir plusieurs à la suite sans utiliser la souris (il est possible de savoir quelle boite à le focus) mais nous n'avons pas reussi
\end{itemize}

\section{Étape 3 : XXXX}

\emph{5 pages}

Description précise de l'extension réalisée - choix éventuels

Description de la résolution (protocole, algorithmes, ... selon l'extension choisie)
Nous utilisons la methode node.js, le seveur execute differentes fonctions selon le format de l'url utilisée pour y acceder et renvoie une chaine de characteres\\

\section{Bilan et Conclusions}

\emph{1 page}

Ce qui a fonctionné, ce qui n'a pas fonctionné, problèmes rencontrés, ...



\end{document}

